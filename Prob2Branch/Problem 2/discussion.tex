\documentclass[10pt]{article}
\usepackage[margin=1in]{geometry}
\usepackage{graphicx}

\begin{document}

\title{CS 131 Finals}
\author{Gerard Arel H. Latoga\\ Faith Therese F. Pena}
\maketitle

\begin{enumerate}
    \item
        \textbf{Does your graph resemble figure 3? (not pictured)}\\
        Yes, the output data can be graphed similarly to figure 3.
    \item
        \textbf{What is the difference, if any?}\\
        The step size $h$ will produce discrepancies in the values of theta.
        But with $h$ small enough, the discrepancies can be insignificant.
    \item
        \textbf{Can you explain what's happening to the pendulum in relation to what your plot is showing?}\\
        As the pendulum swings, it will slow down and have a lower swing due to the effects of gravity countering its upward force.
        This results in having a smaller angle of swing, which the plot shows.
\end{enumerate}

\end{document}
